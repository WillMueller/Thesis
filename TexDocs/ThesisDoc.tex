\documentclass{article}
\usepackage{amsmath}
\usepackage{amsfonts}
\usepackage{array}
\usepackage{dsfont}
\usepackage{hyperref}
\usepackage{amssymb}
\usepackage{amsthm}
\usepackage{amsfonts}
\usepackage{graphicx}
% \usepackage{geometry}
\usepackage{bbold}
\usepackage{caption}
% \usepackage{pseudocode}
\usepackage{standalone}
% \usepackage[nottoc]{tocbibind}
% \usepackage{apacite}
\usepackage{etoolbox}
% \usepackage{tikz}
% \usepackage{color}
% \usepackage[T1]{fontenc}
% \usepackage{lmodern}
% \usepackage{mathptmx}
% \usepackage{standalone}
\usepackage{accents}
\usepackage{enumitem}

% used for typesetting theorem environments
\usepackage{amsthm}

% used for typesetting tikz graphs
\usepackage{tikz}
% used for typesetting automata using tikz
\usetikzlibrary{automata}


% aaron edit

%wills stuff


\usepackage{tikz}
\usepackage{algorithmic}
\usepackage{algorithm}
\usepackage{verbatim}

% \graphicspath{ {../images/} }

\newcommand{\R}{\mathds{R}}
\newcommand{\Zplus}{\mathds{Z}_+}
\newcommand{\N}{\mathds{N}}
\newcommand{\seq}[2][j]{\left\{#2\right\}_{#1=0}^\infty}
\newcommand{\diff}[3][]{\frac{\mathrm{d}^{#1} #2}{\mathrm{d}^{#1} #3}}
\newcommand{\pdiff}[3][]{\frac{\partial^{#1} #2}{\partial^{#1} #3}}
\newcommand{\dd}{\mathrm{d}}
\newcommand{\ip}[2]{\left\langle #1,#2\right\rangle}
\newcommand{\kernl}[1]{\text{ker}(#1)}
\newcommand{\ind}[1]{\mathbb{1}_{#1}~}
\newcommand{\s}{\mathcal{S}}
\newcommand{\Sin}[1]{\sin\left(#1\right)}
\newcommand{\Cos}[1]{\cos\left(#1\right)}
\newcommand{\ubar}[1]{\underaccent{\bar}{#1}}

\newtheorem{thm}{Theorem}
\newtheorem{assmp}{Assumption}
\newtheorem{prop}{Proposition}
\newtheorem{lemma}{Lemma}

\newcolumntype{C}[1]{>{\centering\arraybackslash}p{#1}}
\newcolumntype{L}[1]{>{\raggedright\arraybackslash}p{#1}}
\newcolumntype{R}[1]{>{\raggedleft\arraybackslash}p{#1}}

\captionsetup[table]{labelfont=bf}
\captionsetup[figure]{labelfont=bf}

\title{Group 5 Thesis}
\author{Us}

\begin{document}
	\maketitle

	\section{Introduction}
		Imagine an object recorded by a camera and represented in a sequence of 2-dimensional, coloured images. The object may change size due to changes in position (closer to the camera) or even by nature (an inflating balloon). The object may enter a shadow or areas with different lighting causing its brightness and colour to change between images. If the object or the camera moves and rotates, then the location of the object, its shape, and even the background of the object will alter over the sequence of images. Nonetheless, a human could view this sequence of images and associate a region in each image as one object being manipulated. 

		Region tracking in a sequence of images seeks to create a computer program that, given a region in an initial image, can identify the region over subsequent images. The tracking is desired to be achieved even if some characteristics of the region change. This report proposes two distinct applications for such a program: hand-gesture tracking and tracking of anatomical structures of CT/MRI scans. While hand-gesture recognition will process images whose sequence represents change in time, the CT/MRI scan will process images whose sequence represents change along a dimension of the 3-dimensional object that was scanned.

		\subsection{Appliction: Medical Imaging}
			The main imaging modalalities chosen were X-ray Computed Tomography (CT) and Magtetic Resinance Imaging (MRI) scans. These modalities were chosen due to their diagnostic proformance and ubiqity of use. For surgons these scans are an integral portion of planning and preparing for precedures. The ability to automaticaly segment tissues and render models of them in 3D aids in this process and can also be used to produce implants that fit the pateint's antomy to a greater degrree.


		\subsection{Aplication: Gesture Tracking}
			Following a hand through a video sequence can provide idaviduals with a touchless interface like the Kinect. Using the cheap an ubiqitus web cam the ability to follow gestures can allow a user to interact more fluidly with technology and prevent interuption of some tasks.

	\section{Prob Def}
		Produce an algorithm that eficently tracks regions in medical and video sequences with low error.

	\section{Math}
		\subsection{Problem Formulation}

		\subsection{Simple Closed Curves}

		\subsection{Level Sets}

		\subsection{Functional Derivations}

	\section{Design}
		Fasal Aaron

		\subsection{Medical}
			Aaron

		\subsection{Gesture}
			Fasal

	\section{Implementation}
		Zach

	\section{Results}
		Zach

	\section{Discussion (3bottom)}
		Mandy

	\section{Conclusion}
		AAron

	\section{Future Work}
		Faisal

	\bibliographystyle{plain}
	\bibliography{ref-bibtex}
	
\end{document}
